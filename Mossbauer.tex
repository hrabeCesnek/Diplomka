% -----------------------------------------------
% Vlastní text práce (kapitoly práce)
% -----------------------------------------------

% -----------------------------------------------
\chapter{Mössbauer effect}
% -----------------------------------------------
Mössbauer effect is a physical effect, which can under certain circumstances occur on atomic nuclei. It consists of recoilless resonance emission/absorption of gamma photons by nuclei of source/absorber with discrete nuclear energy levels.  

% -----------------------------------------------
\section{Physical concept}
It is well-known fact, that the atomic nuclei are quantum systems with discrete energy levels (analogous to the energy levels of electron shell), thus upon deexcitation or excitation they absorb/emit gamma photon with energy equal to the difference between the levels. However, this energy may be altered - due to the  momentum conservation law, some part of the gamma photon energy is transferred to the  kinetic energy of nucleus, crystal as whole body or is possibly transformed into lattice vibrations (phonons). Due to this fact, the emission and absorption energy spectra may be different and without any overlaps, which prevents the resonance effects from happening.

% -----------------------------------------------

\section{Mössbauer Fe$^{57}$ spectroscopy}


This thesis is mainly devoted to the application of semiconductor detectors for detection of the 14.4 keV gamma photons.

It is also necessary to consider, that the Fe$^{57}$ isotope have relative abundance only $2.21 \%$ \cite{compounds}. Although this fact, the spectra are still be measured with very respectable precision and efficiency, which makes the Mössbauer spectroscopy very sensitive measurement method.
% %%%%%%%%%%%%%%%%%%%%%%%% End of file %%%%%%%%%%%%%%%%%%%%%%%%
