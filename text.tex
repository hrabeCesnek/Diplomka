% -----------------------------------------------
% Vlastní text práce (kapitoly práce)
% -----------------------------------------------

% -----------------------------------------------
\chapter{Název kapitoly}
% -----------------------------------------------
Lorem ipsum dolor sit amet, consectetur adipiscing elit. Curabitur et lectus sit amet lectus vestibulum dignissim. Cras sit amet enim vitae mi elementum blandit eget nec tortor. Curabitur eget eros vitae arcu luctus varius commodo vel mauris. Nam elementum convallis pretium. Nunc dignissim pulvinar urna, nec blandit ante fringilla at. Ut et magna purus, vel pellentesque massa. In tortor nisi, faucibus condimentum cursus ut, sollicitudin quis leo. Ut at purus nec arcu accumsan tincidunt id id massa. Nam id vehicula mi. 

% -----------------------------------------------
\section{Název podkapitoly}
% -----------------------------------------------
Lorem ipsum dolor sit amet, consectetur adipiscing elit. Curabitur et lectus sit amet lectus vestibulum dignissim. Cras sit amet enim vitae mi elementum blandit eget nec tortor. Curabitur eget eros vitae arcu luctus varius commodo vel mauris. Nam elementum convallis pretium. Nunc dignissim pulvinar urna, nec blandit ante fringilla at. Ut et magna purus, vel pellentesque massa. In tortor nisi, faucibus condimentum cursus ut, sollicitudin quis leo. Ut at purus nec arcu accumsan tincidunt id id massa. Nam id vehicula mi. 
% -----------------------------------------------
\begin{equation}
  \label{eq:rovnice1}
  \int\limits_0^{\infty}\bgomega{\rm d}t = 1,234\,\varkappa\vec{A}.
\end{equation}
% -----------------------------------------------
Podle rovnice~(\ref{eq:rovnice1}), jak je uvedeno v~\cite{gravitation}. 

% -----------------------------------------------
\section{Název další podkapitoly}
% -----------------------------------------------
Další příklady matematické sazby:
% -----------------------------------------------
\begin{gather}
a_1=b_1+c_1\\
a_2=b_2+c_2-d_2+e_2
\end{gather}
% -----------------------------------------------
\ldots
% -----------------------------------------------
\begin{align}
a_{11}& =b_{11}&
a_{12}& =b_{12}\\
a_{21}& =b_{21}&
a_{22}& =b_{22}+c_{22}
\end{align}
% -----------------------------------------------
\ldots
% -----------------------------------------------
\begin{equation}\label{first}
a=b+c
\end{equation}
% -----------------------------------------------
\ldots
% -----------------------------------------------
\begin{subequations}\label{grp}
\begin{align}
a&=b+c\label{second}\\
d&=e+f+g\label{third}\\
h&=i+j\label{fourth}
\end{align}
\end{subequations}
% -----------------------------------------------
\ldots z~rovnice (\ref{second})
% -----------------------------------------------
\[
A_\infty + \pi A_0
\sim \mathbf{A}_{\boldsymbol{\infty}} \boldsymbol{+}
\boldsymbol{\pi} \mathbf{A}_{\boldsymbol{0}}
\sim\pmb{A}_{\pmb{\infty}} \pmb{+}\pmb{\pi} \pmb{A}_{\pmb{0}}
\]
% -----------------------------------------------
\ldots
% -----------------------------------------------
\[
\begin{pmatrix}
\alpha& \beta^{*}\\
\gamma^{*}& \delta
\end{pmatrix}\qquad
P_{r-j}=\begin{cases}
0& \text{když $r-j$ je liché},\\
r!\,(-1)^{(r-j)/2}& \text{když $r-j$ je sudé}.
\end{cases}
\]
% -----------------------------------------------
\ldots
% -----------------------------------------------
\begin{equation}
\Re{z} =\frac{n\pi \dfrac{\theta +\psi}{2}}{
\left(\dfrac{\theta +\psi}{2}\right)^2 + \left( \dfrac{1}{2}
\log \left\lvert\dfrac{B}{A}\right\rvert\right)^2}.
\end{equation}
% -----------------------------------------------
% %%%%%%%%%%%%%%%%%%%%%%%% End of file %%%%%%%%%%%%%%%%%%%%%%%%
